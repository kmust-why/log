% LaTex文档的基本结构、编译和调试、命令符号的输入(如%,{..}等等)
\documentclass{book} %book,report,letter
% 在\documentclass{artical}和begin{document}之间的部分被称为导盲区
%\usepackage{amsmath} %宏包,美国数学协会
\usepackage{enumerate}

\begin{document}
% 正文部分
\begin{itemize}
	\item this is item1
	\begin{itemize}
		\item this is item1
		\begin{itemize}
			\item this is item1
			\begin{itemize}
				\item this is item1
				\item this is item2
				\item this is item3
			\end{itemize}
			\item this is item2
			\item this is item3
		\end{itemize}
		\item this is item2
		\item this is item3
	\end{itemize}
	\item this is item2
	\item this is item3
\end{itemize}

\begin{enumerate}
	\item this is item1
	\begin{enumerate}
		\item this is item1
		\begin{enumerate}
			\item this is item1
			\item this is item2
			\item this is item3 
		\end{enumerate}
		\item this is item2
		\item this is item3 
	\end{enumerate}
	\item this is item2
	\item this is item3 
\end{enumerate}

\begin{enumerate}[\bfseries A.]
\setcounter{enumi}{4}
	\item this is item1
	\begin{enumerate}[\sffamily a.]
		\item this is item1
		\begin{enumerate}[i.]
			\item this is item1
			\item this is item2
			\item this is item3 
		\end{enumerate}
		\item this is item2
		\item this is item3 
	\end{enumerate}
	\item this is item2
	\item this is item3 
\end{enumerate}

\begin{tabular}{clr}
223&112&333\\
23&12&33\\
\end{tabular}

\begin{tabular}{|c|l|r|}
\hline
223&112&333\\
\hline
23&12&33\\
\hline
\end{tabular}

\end{document}

